\documentclass[12pt]{article}

\usepackage[catalan]{babel}
\usepackage[utf8]{inputenc}
\usepackage[T1]{fontenc}
\usepackage{lmodern}
\usepackage[a4paper,total={6.2in,9.3in}]{geometry}
% Formatting options
\usepackage[bf,sf]{titlesec} % Make the section titles bold and sans-serif
\usepackage[font={footnotesize,sf}]{caption} % Make captions small and sans-serif
\renewcommand{\arraystretch}{1.7}
\usepackage{amsmath,amssymb}

\newcommand{\abs}[1]{\left\lvert#1\right\rvert}
\newcommand{\R}{\mathbb{R}}

\newcommand{\yestag}{\refstepcounter{equation}\tag{\theequation}}

\title{\textsf{\textbf{Mètodes Numèrics \\ Pràctica 3:} Interpolació}}
\author{\textsf{Raquel Garcia, Arnau Mas}}
\date{\textsf{}}

\begin{document}
	\maketitle
	\section*{Problema 2}
	Considerem la funció de Bessel de primera espècie d'ordre zero, $J_0(x)$. Volem estimar el valor en l'absissa $x^*$ tal que $J_0(x^*)=0$, per això realitzarem interpolació inversa de graus $1$, $3$ i $5$, és a dir, si $(x, J_0(x))$ són els punts de la funció de Bessel que ens dónen, nosaltres prendrem $(J_0(x), x)$ com a punts per a realitzar la interpolació. De manera que si $p(x)$ és el nostre polinomi interpolador, al avaluar $p(0)$ obtindrem una aproximació de $x^*$.\\
	
	El programa \texttt{prob2.c} donats uns punts com a entrada, calcula el polinomi interpolador de Lagrange amb el mètode de les diferències dividides de Newton i l'avalua en el punt zero amb la regla de Horner. Els resultats que hem obtingut al realitzar les interpolacions demanades han sigut els següents:
	
	\begin{table}[h!]
		\centering
		\caption{Interpolant valors positius de $J_0(x)$ més pròxims al canvi de signe de la funció.}	
		\begin{tabular}{c|c|c}
			Grau & $x^*$ &$J_0(x^*)$\\
			\hline
			\hline
			$1$ & $2.404728613882804$  &$5.03291522479392\times10^{-5}$\\
			$3$ & $2.404822718113948$ &$1.47416266795862\times10^{-6}$\\
			$5$ & $2.404825294785460$ &$1.36489238329446\times10^{-7}$\\
		\end{tabular}
	\end{table}
	
	\begin{table}[h!]
		\centering
		\caption{Interpolant valors negatius de $J_0(x)$ més pròxims al canvi de signe de la funció.}	
		\begin{tabular}{c|c|c}
			Grau & $x^*$ &$J_0(x^*)$\\
			\hline
			\hline
			$1$ & $2.400077241947102$&$2.46750381340249\times10^{-3}$\\
			$3$ & $2.404149375353531$&$3.51087705194527\times10^{-4}$\\
			$5$ &$2.404216734868258$ &$3.16108843546827\times10^{-4}$\\
		\end{tabular}
	\end{table}
	
	\begin{table}[h!]
		\centering
		\caption{Interpolant valors de $J_0(x)$ simètrics al canvi de signe de la funció.}	
		\begin{tabular}{c|c|c}
			Grau & $x^*$ &$J_0(x^*)$\\
			\hline
			\hline
			$1$ & $2.404927513002775$  &$-5.29287203952310\times10^{-5}$\\
			$3$ & $2.404824021911155$ &$7.97298995297100\times10^{-7}$\\
			$5$ & $2.404825653043717$ &$-4.94996456864148\times10^{-8}
			$\\
		\end{tabular}
	\end{table}
	Observem que el millor resultat l'obtenim amb la interpolació de grau 5 de valors simètrics al voltant del canvi de signe de $J_0(x)$. En general les millors són les interpolacions amb els valors positius i amb els simètrics, i la interpolació amb valors negatius és prou dolenta en comparació.\\
	
	De fet ja podíem esperar des del principi que la millor interpolació fos la simètrica, ja que és l'única que en l'interval de punts interpoladors conté al $x^*$, i en general si tenim els punts interpoladors en un interval $[a,b]$ i volem avaluar el polinomi interpolador $p(x)$ en un punt fora d'aquest interval, l'error obtingut pot ser molt gran.
\end{document}
