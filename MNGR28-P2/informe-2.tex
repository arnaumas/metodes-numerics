\documentclass[12pt]{article}

\usepackage[catalan]{babel}
\usepackage[utf8]{inputenc}
\usepackage[T1]{fontenc}
\usepackage{lmodern}
\usepackage[a4paper,total={6.2in,9.3in}]{geometry}
% Formatting options
\usepackage[bf,sf]{titlesec} % Make the section titles bold and sans-serif
\usepackage[font={footnotesize, sf}, labelfont=bf]{caption} % Format dels peus de figura
\renewcommand{\arraystretch}{1.7}
\usepackage{amsmath,amssymb}
\usepackage{siunitx}
\usepackage{booktabs}

\newcommand{\abs}[1]{\left\lvert#1\right\rvert}
\newcommand{\R}{\mathbb{R}}
\newcommand{\N}{\mathbb{N}}

\newcommand{\yestag}{\refstepcounter{equation}\tag{\theequation}}

\title{\textsf{\textbf{Mètodes Numèrics \\ Pràctica 2:} Zeros de funcions}}
\author{\textsf{Raquel Garcia, Arnau Mas}}
\date{\textsf{11 de Març 2018}}

\begin{document}
\maketitle
\section*{Problema 1}
Considerem l'equació polinòmica
\begin{equation}\label{pol:1}
	x^3=x+40.
\end{equation}
UtIlitzant les fórmules de Cardano trobem l'arrel $\alpha$ que ve donada per:
\begin{equation}\label{alfa}
	\alpha=\left(20+\frac{1}{9}\sqrt{32397} \right)^{1/3}+\left(20-\frac{1}{9}\sqrt{32397} \right)^{1/3}
\end{equation}
Tot i que ens dóna l'arrel exacta, aquesta expressió no és bona des del punt de vista numèric, ja que al segon terme hi ha una resta que produeix cancelació. El programa \texttt{prob1a.c} avalua aquesta expressió en doble i en simple precisió. En precisió doble obtenim $\alpha=\num{3.517393514052852}$ i en precisió simple $\alpha=\num{3.51738477}$, mentre que el resultat exacte amb 15 decimals és $\alpha=\num{3.517393514052818}$. Per tant en precisió doble, l'error relatiu que s'ha produït ha sigut $\varepsilon_d=\num{9.67d-15}$, i en simple $\varepsilon_s=\num{2.51d-6}$. Ara aplicarem la fórmula de propagació de l'error relatiu al segon terme de \eqref{alfa} (sense l'arrel cúbica) per a provar d'estimar aquest error:
\begin{equation}\label{errorrelatiu1}
	\varepsilon_r\left( 20-\frac{1}{9}\sqrt{32397}\right) =\dfrac{20+\frac{1}{9}\sqrt{32397}}{ 20-\frac{1}{9}\sqrt{32397}}\epsilon_r
\end{equation}
On $\epsilon_r$ és l'error relatiu que suposarem prové només de l'expressió en punt flotant i que és el mateix per als dos sumands, de l'ordre de $\num{d-17}$ en precisió doble, i de $\num{d-10}$ en precisió simple. D'aquesta manera obtenim una estimació de $\varepsilon_r(\alpha)\sim \num{d-13}$ en precisió doble  i $\varepsilon_r(\alpha)\sim \num{d-6}$ en precisió simple, on també hem suposat que la resta d'operacions que es realitzen en l'avaluació d'$\alpha$ no modifiquen de manera significativa l'ordre d'aquests errors relatius.

A continuació utilitzarem el mètode de Newton per a resoldre \eqref{pol:1}, amb $f(x)=x^3-x-40$, i com a punt inicial $x_0=2$. El programa \texttt{prob1b\_do.c} executa el mètode en precisió doble, i \texttt{prob1b\_fl.c} l'executa en precisió simple. En doble obtenim $\alpha=\num{3.517393514052818}$ després de $7$ iteracions, i en simple obtenim $\alpha=\num{3.51739359}$ després de $5$ iteracions. Si considerem el mètode de Newton com un mètode del punt fix amb funció d'iteració

\begin{equation}\label{pfix}
	g(x)=x-\dfrac{x^3-x-40}{6x^2-1}
\end{equation}
tenim que $|g'(2)|\approx3.37>1$, per tant per a aquest $x_0$ no tenim clar si convergirà, ni podem fer una estimació a priori del nombre d'iteracions necessàries. El valor que obtenim després de fer una iteració del mètode de Newton és $x_1\approx5.091$, per a aquest valor $|g'(x_1)|\approx0.45$, que és molt millor que $x_0$ i ens permet estimar a priori que el nombre d'iteracions necessàries serà com a màxim de
\begin{equation}\label{iteracions}
	n\sim\left( \dfrac{\log(\varepsilon(1-0.45)/|5.091-2|)}{\log(0.45)}\right)+1,
\end{equation}
on $\varepsilon$ és la fita de l'error que volem aconseguir. Per a tenir $15$ decimals correctes, és a dir $\varepsilon<10^{-15}$ a priori necessitem $n\sim47$, i per a tenir $8$ decimals, $\varepsilon<10^{-8}$ a priori necessitem $n\sim27$ iteracions. Aquestes fites són tan elevades degut a que el punt $x_0=2$ és un punt molt dolent on començar.


Considerem ara l'equació polinòmica
\begin{equation}\label{pol:2}
	x^3=x+400.
\end{equation}
Si escrivim aquesta equació com a $p(x)=x^3-x-400=0$, veiem que només hi ha un canvi de signe en els seus coeficients, i per tant per la regla dels signes de Descartes tenim que aquest polinomi té una única arrel positiva. Com que $p(2)=-394<0$, $p(8)=104>0$ i $p(x)$ és continu, pel teorema de Bolzano tenim que aquesta arrel $\beta$ es troba a l'interval $[2,8]$. La fórmula de Cardano que obtenim per a aquesta arrel és
\begin{equation}\label{betaarrel}
	\beta=\left( 200+\frac{1}{9}\sqrt{3239997}\right)^{1/3}\left( 200-\frac{1}{9}\sqrt{3239997}\right)^{1/3}.
\end{equation}
El programa \texttt{prob1c.c} avalua $\beta$ en precisió doble. Obtenim $\beta=\num{7.413302725859884}$, però el valor veritable amb $15$ xifres decimals és $\beta=\num{7.413302725857898}$, per tant es produeix un error absolut $\varepsilon_a(\beta)=1.986\times10^{-12}$. Aquest error prové fonamentalment de la cancelació que es produeix al segon terme de \eqref{betaarrel}.

Fent servir el programa \texttt{prob1c123.c} obtenim $15$ decimals correctes de $\beta$, treballant amb precisió doble, aplicant el mètode de la bisecció i el mètode de la secant partint de l'interval $[2,8]$, i el mètode de Newton amb $x_0=2$.

Amb el mètode de la bisecció hem necessitat $50$ iteracions per a trobar $\beta$, amb el de la secant hem necessitat $8$ iteracions, mentre que amb el de Newton n'hem necessitat $10$. Observem que el mètode de la bisecció és molt lent comparat amb els altres dos, però el mètode de Newton i el de la secant tampoc han sigut especialment ràpids degut a que el punt $x_0=2$ es troba prou allunyat de l'arrel.

\newpage

\section*{Problema 2}
El programa \texttt{prob2a.c} conté codi que executa la iteració descrita al problema 2. Si comencem la iteració a \( x_0 = \num{7.5} \) aleshores obtenim l'arrel \( x^* \) amb 15 xifres decimals correctes després de 4 iteracions. Obtenim \( x^* = \num{7.413302725857898} \), que és coherent amb els resultats del problema anterior.  

Per estudiar l'ordre de convergència aproximarem \( \abs{x_k - x^*} \) per \( e_k = \abs{x_{k} - x_{k+1}} \). El codi de \texttt{prob2a.c} també realitza el càlcul de \( e_k/e_{k-1} \), \( e_k/e_{k-1}^2 \) i \( e_k/e_{k-1}^3 \). Com que després de 4 iteracions ja hem obtingut \( x^* \) amb més precisió que la que permet el format \texttt{double} \( e_5 \) és 0. Per tant el quocient \( e_6 / e_5 \) dóna \texttt{nan} com a resultat. Tot i això, només amb 4 iteracions ja podem dir que l'ordre de convergència és quadràtic.  

\begin{table}[ht]
	\sffamily \small
	\centering
	\caption{Anàlisi de l'ordre convergència amb pivot inicial \num{7.5}}
	\label{tab:ordre 2}
	\begin{tabular}{SS[table-parse-only]SS[table-parse-only]}
		\toprule
		{ \( k \) } & { \( \frac{e_k}{e_{k-1}} \)} & { \( \frac{e_k}{\left(e_{k-1}\right)^2} \)} & { \( \frac{e_k}{\left(e_{k-1}\right)^3} \)} \\
		\midrule
		1 & 0.0114 & 0.1336 & 1.5993 \\
		2 & 0.0235 & 23.9687 & 24428 \\
		3 & 0.0054 & 23.3059 & d6 \\
		4 & d-7 & 23.0688 & d9 \\
		\bottomrule
	\end{tabular}
\end{table}

Efectivament, tal i com mostra la taula \ref{tab:ordre 2}, l'únic quocient que es manté estable és \( \frac{e_k}{\left(e_{k-1}\right)^2} \). A més també observem que \( \frac{e_k}{\left(e_{k-1}\right)^3} \to \infty \) i \( \frac{e_k}{e_{k-1}} \to 0 \) la qual cosa ens porta a concloure que l'ordre de convergència és exactament quadràtic.
\newpage

\section*{Problema 3}
Considerem l'equació $f(x)=0$, amb $f(x)$ continuament derivable. Si $x^*$ és una arrel simple, de manera que compleix $f(x^*)=0$ i $f'(x)\neq0$ en un entorn de $x^*$, aleshores podem utilitzar el métode de Halley que consisteix en la iteració
\begin{equation}\label{halley}
	x_{k+1}=x_k-\dfrac{2f(x_k)f'(x_k)}{2(f'(x_k))^2-f(x_k)f''(x_k)}
\end{equation}
per a aproximar $x^*$.

Al programa \texttt{prob3.c} hem utilitzat aquest mètode per a calcular l'arrel de $f(x)=x^3-x-400$ amb $15$ decimals correctes. Hem obtingut $x^*=\num{7.413302725857898}$ en $5$ iteracions, partint de $x_0=2$.

Per a comprovar que aquest mètode té ordre de convergència $3$ considerem $e_k=|x_k-x_{k-1}|$ i estudiem els quocients $\frac{e_k}{(e_{k-1})^3}$. Els resultats es mostren a la taula \ref{tab:ordre 3}.

\begin{table}[ht]
	\sffamily \small
	\centering
	\caption{Anàlisi de l'ordre de convergència del mètode de Halley}
	\label{tab:ordre 3}
	\begin{tabular}{SSS}
		\toprule
		{ \( k \) } & { \(x_k \)} & { \( \frac{e_k}{\left(e_{k-1}\right)^3} \)} \\
		\midrule
		1 & 3.744 & {---} \\
		2 & 6.305 & 0.4828 \\
		3 & 7.392 & 0.0647 \\
		4 & 7.413 & 0.0163 \\
		5 & 7.413 & 0.0124 \\
		\bottomrule
	\end{tabular}
\end{table}
Veiem que $e_k\sim (e_{k-1})^3$, de manera que el mètode de Halley té ordre de convergència cúbic.

\newpage

\section*{Problema 4}
El programa \texttt{prob4.c} conté codi que executa la iteració proposada. Com que sabem a quin valor ha de convergir la iteració podem analitzar directament l'error absolut \( e_k = \abs{p_k - \pi} \). Aquest decreix fins a la quarta iteració, però es compleix \( e_5 > e_4 \). A partir d'aquí perdem la convergència ja que l'error absolut comença a creixer. Si analitzem l'algoritme sembla que l'error ha d'aparèixer a \( c_k \) o bé a \( s_k \), ja que per calcular-les s'ha de fer una resta, la qual cosa pot introduir errors de cancel·lació. Quan observem el valor de \( c_k \) i \( s_k \) a cada iteració veiem que a partir de la quarta iteració \( c_k \) és 0 i que, per tant, \( s_k \) es manté constant. Això té sentit ja que de fet \( c_k \to 0 \). 

La desigualtat aritmètica-geomètrica ens dóna \( a_k > b_k \). Per tant \( a_{k+1} = \frac{a_k + b_k}{2} < a_k \) i tenim que la successió \( a_k \) és decreixent i fitada inferiorment ---per \( b_0 \), per exemple--- i per tant convergent. Similarment, \( b_{k+1} = \sqrt{b_k a_k} > b_k \) ja que \( a_k > b_k \). Per tant, com que \( b_k \) és una successió creixent i fitada superiorment---per exemple per \( a_0 \)--- aleshores és convergent. Posem \( a_k \to \alpha \) i \( b_k \to \beta \). Aleshores \( \alpha \geq \beta \). I si \( \alpha > \beta \) aleshores \( \alpha - \beta > 0 \). En particular, existeix \( K \in \N \) tal que \( a_K - \alpha < \frac{\alpha - \beta}{2} \) i tal que \( \beta - b_K < \frac{\alpha - \beta}{2} \). Per tant \( a_K + b_K < \alpha + \beta \) i \( a_{K+1} = \frac{a_K + b_K}{2} < \frac{\alpha + \beta}{2} < \alpha \), que és una contradicció ja que sabem que \( a_k \) decreix cap a \( \alpha \). Per tant \( \alpha = \beta \), que implica \( c_k \to 0 \). En particular, a partir d'una iteració \( N \) tindrem que \( c_k < \epsilon \) per \( k \geq N \) amb \( \epsilon \) l'èpsilon màquina. I per tant \( \mathsf{fl}(s_{k+1}) = \mathsf{fl}(s_k) \) a partir d'aquesta iteració. Això és precisament el que observem a partir de la quarta iteració. 

Pel que fa a l'ordre de convergència, si apliquem el mateix mètode que al problema 2, amb l'avantatge de què coneixem el límit de la iteració i per tant \( e_k = \abs{p_k - \pi} \) trobem que només el quocient \( e_k/\left(e_{k+1}\right)^2 \) es manté estable, tal i com es comprova a la taula \ref{tab:ordre 4} i concloem que la convergència és quadràtica.

\begin{table}[ht]
	\sffamily \small
	\centering
	\caption{Anàlisi de l'ordre convergència}
	\label{tab:ordre 4}
	\begin{tabular}{SS[table-parse-only]SS[table-parse-only]}
		\toprule
		{ \( k \) } & { \( \frac{e_k}{e_{k-1}} \)} & { \( \frac{e_k}{\left(e_{k-1}\right)^2} \)} & { \( \frac{e_k}{\left(e_{k-1}\right)^3} \)} \\
		\midrule
		1 & 0.0537 & 0.0625 & 0.0729 \\
		2 & 0.0019 & 0.0413 & 0.8957 \\
		3 & d-6 & 0.0398 & 454 \\
		\bottomrule
	\end{tabular}
\end{table}
\newpage

\section*{Problema 5}
El nostre objectiu és obtenir una aproximació de l'arrel quadrada d'un nombre utilitzant l'expressió
\begin{equation*}
	\sqrt{1+x}=f(x)\sqrt{1+g(x)}
\end{equation*}
on $g$ és un infinitèsim d'ordre més petit que $x$ per a $x$ tendint a $0$. Si triem $f(x)$ com una aproximació de $\sqrt{1+x}$ aleshores es pot calcular $g(x)$ com
\begin{equation}\label{g(x)}
	g(x)=\dfrac{1+x}{f(x)^2}-1.
\end{equation}
Triarem $f(x)$ com una funció racional $p(x)/q(x)$ de manera que $p$ i $q$ tenen el mateix grau i el seu desenvolupament de Taylor coincideix amb el de $\sqrt{1+x}$ fins a un cert grau.

Si volem que $p$ i $q$ siguin lineals, aleshores el desenvolupament de $p(x)-q(x)\sqrt{1+x}$ ha de tenir els tres primers termes nuls, i es diu que $f$ és una aproximació de Padé de la funció $\sqrt{1+x}$. Posem $p(x)=a_1x+a_0$ i $q(x)=b_1x+b_0$ de manera que s'ha de complir
\begin{equation}\label{pade}
	a_1x+a_0-\left( 1+\frac{1}{2}x-\frac{1}{8}x^2\right) \left(b_1x+b_0\right)=0,
\end{equation}
i trobem
\begin{equation*}
	f_1(x)=\dfrac{3x+4}{x+4}.
\end{equation*}
Per a aquesta $f_1(x)$ tenim que $g_1(x)$ és:
\begin{equation} \label{eq:g1}
	g_1(x)=\dfrac{x^3}{9x^2+24x+16} = \frac{x^3}{(3x + 4)^2}. 
\end{equation}

Considerem ara la succesió $a_0=x$, $a_{n+1}=g(a_n)$ i $b_n=f(a_n)$. Volem comprovar que
\begin{equation}\label{volem demostrar}
	\sqrt{1+x}=\left(\prod_{j=0}^kb_j \right)\sqrt{1+a_{k+1}}.
\end{equation}
Ho veurem per inducció sobre $k$. Per a $k=0$ tenim
\begin{equation*}
	\sqrt{1+x}=f(x)\sqrt{1+g(x)}
\end{equation*}
que és cert ja que és l'hipòtesi inicial. Per tot \( k \in \N \) tenim que
\begin{equation}\label{blabla}
	a_{k+1}= \frac{1 + a_k}{b_{k+1}^2} - 1
\end{equation}
a partir de \ref{g(x)}. Aleshores, fent servir la hipòtesi d'inducció tenim  
\begin{equation*}
	\left(\prod_{j=0}^{k+1} b_j\right) \sqrt{1 + a_{k+1}} = \left(\prod_{j=0}^{k+1} b_j\right) \sqrt{\frac{1+a_k}{b_k^2}} = \left(\prod_{j=0}^{k} b_j\right) \sqrt{1 + a_k} = \sqrt{1+x},
\end{equation*}
que conclou la prova. 

A partir de la forma explícita per a \( g_1 \) que hem trobat a \ref{eq:g1}, havent fet la tria de \( f_1 \) com a quocient de polinomis de grau 1 és clar que \( g_1(x) = O(x) \) ja que \( g_1(x) / x \to \frac{1}{9} \) quan \( x \to 0 \). A més tenim
\begin{equation*}
	g'_3(x) = \frac{3x^2(3x+4)^2 - 6x^3(3x+4)}{(3x+4)^4} = \frac{3x^3 + 12x^2}{(3x+4)^3} < \frac{1}{9}	
\end{equation*}
per \( x > 0 \). Per tant, pel teorema del valor mitjà, per \( x>0 \) \( g_3 \) és Lipschitz amb constant \( \frac{1}{9} \) i per tant contractiva. A més, com que \( g_3(0) = 0 \), \( \abs{g_3(x)} < \abs{x} \) per \( x > 0 \). 

Si volem que $f(x)$ sigui quocient de polinomis de grau 3, podem fer una cosa semblant a \eqref{pade}, ja que sempre podrem aconseguir equacions lineals suficients. Nosaltres en aquest cas hem utilitzat SageMath per a trobar l'aproximació de Padé amb polinomis de tercer grau de $\sqrt{1+x}$, i hem obtingut
\begin{equation*}
	f_3(x)=\dfrac{7x^3+56x^2+112x+64}{x^3+24x^2+80x+64}.
\end{equation*}
Per a aquesta $f_3(x)$ tenim que $g_3(x)$ és:
\begin{equation}\label{g3x}
	g_3(x)=\dfrac{x^7}{49x^6+784x^5+4704x^4+13440x^3+19712x^2+14336x+4096}.
\end{equation}

A continuació volem comprovar la desigualtat
\begin{equation}\label{desigualtat}
	\abs{\sqrt{1+x}-\prod_{j=0}^{k}b_j}\leq\frac{a_{k+1}}{2}\sqrt{1+x}.
\end{equation}
Observem que aïllant $\prod b_j$ de \eqref{volem demostrar} i substituïnt-lo en \eqref{desigualtat}, obtenim que la desigualtat que volem provar és equivalent a
\begin{equation}
	\left| 1-\dfrac{1}{\sqrt{1+a_{k+1}}}\right|\leq\dfrac{a_{k+1}}{2}.
\end{equation}
Podem eliminar el valor absolut ja que \( \sqrt{1 + a_k} > 1 \) i per tant $1/\sqrt{1+a_{k+1}}<1$. Per tant la desigualtat és equivalent a
\begin{equation}
	1-\dfrac{a_{k+1}}{2}\leq\dfrac{1}{\sqrt{1+a_{k+1}}}.
\end{equation}
Com que el desenvolupament de Taylor de $1/\sqrt{1+x}$ fins a grau 2 al voltant de 0 és
\begin{equation}
	\dfrac{1}{\sqrt{1+x}}=1-\dfrac{x}{2}+\frac{3}{8}x^2 + O(x^3).
\end{equation}
Per tant, el residu de Lagrange és \( \frac{3}{8}c^2 \) amb \( c \in [0, a_{k+1}] \), i per tant sempre positiu.

Per a $f_3(x)$, el programa \texttt{probextrah.c} calcula els valors que pren la succesió $a_{n+1}=g(a_n)$ amb $a_0=1$, d'aquesta manera obtenim que \( a_3 = \num{4.537551d-261} \), i per tant observem que es verificarà la desigualtat
	\begin{equation}
		|\sqrt{2}-b_0b_1b_2|\leq4.537551\times10^{-261}\frac{\sqrt{2}}{2}
	\end{equation}
	La conclusió que podem prendre d'aquest fet, és que $\prod_{j=0}^{k}b_j$ amb $b_n=f(a_n)$ és una molt bona aproximació de l'arrel quadrada d'un nombre $a_0+1$, ja que per ser $g(x)$ contractiva i tenir com a únic punt fix $g(0)=0$, la succesió $a_n$ tendeix a $0$ per a $n\to\infty$, i per tant $\prod_{j=0}^{k}b_j\to\sqrt{1+a_0}$. De fet veiem que per a valors petits de $a_0$ aquesta convergència és molt ràpida. Amb aquestes succesions podriem dissenyar un programa per a calcular arrels quadrades de nombres majors que $1$ ($a_0\geq0$): només cal tenir guardades les funcions $f(x)$ i $g(x)$ i executar les succesions anteriors on només intervenen productes, sumes i quocients, que són operacions segures des del punt de vista numèric. 
	
	Al programa \texttt{probextrah.c} hem programat aquestes succesions, utilitzant $f$ com a producte de polinomis d'ordre $3$ i també d'ordre $1$. Per exemple al calcular $\sqrt{2}$ hem obtingut $\sqrt{2}=\num{1.414214553395256}$ en $2$ iteracions amb $f_3$. I fent servir \( f_1 \) obtenim $\sqrt{2}=\num{1.414213562373095}$ en $3$ iteracions. Observem que l'error és major utilitzant polinomis de grau $3$! En teoria $f_3(x)$ hauria de ser molt més efectiva, ja que l'error mesurat amb $g_3(x)$ tendeix a zero molt més ràpidament. Però de fet, aquest és el problema. Si ens fixem en $f_3(x)$, veiem que per a $x$ molt petita ens donarà $64/64=1$ com a resultat, ja que si operem amb precisió doble, \( g(a_k) \) serà menor que l'èpsilon màquina molt depressa. Així veiem que amb $f_3(x)$ tenim $a_1= \num{1.890824d-5}$ i $a_2= \num{2.109448d-37}$, i només és útil $a_1$ ja que \( a_2 \) ja és menor que l'èpsilon màquina. En canvi, amb $f_1(x)$ tenim $a_1=\num{2.040816d-2}$, $a_2=\num{5.153446d-7}$ i $a_3=i\num{8.554074d-21}$. Ara \( a_3 \) és menor que l'èpsilon màquina, però amb dues iteracions ja hem comprovat que l'error, almenys per nombres propers a 1, és menor que la precisió del format \texttt{double}.

\end{document}
